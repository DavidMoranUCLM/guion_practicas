\chapter{Resultados y discusión}
\label{ch:resultados}

\avisoLocalizacionArchivo 

Escribe en este capítulo los resultados del proyecto.  Este capítulo debería explicar los resultados de forma global, no los resultados de cada fase o iteración.  Probablemente será el capítulo con más tablas y gráficas.  Revisa las secciones~\ref{sec:figuras} y~\ref{sec:tablas} para aprender cómo se escriben en \LaTeX{}.

Tus contribuciones no tienen por qué limitarse al trabajo sistemático del TFG.  Puede que hayas contribuido en aspectos metodológicos, en ideas novedosas, en la planificación de experimentos, en desarrollos matemáticos. Este capítulo está para agrupar todo eso.  Describe con claridad todo lo que ha supuesto contribuciones originales por tu parte.

\warning{Es importante destacar que un resultado negativo es también un resultado.  Es posible que el proyecto planteara abordar un problema con un método que ha demostrado ser no apto.  Si el trabajo ha sido sistemático sigue teniendo mucho valor, puesto que excluye el método para cualquier otro trabajo futuro.  Escribe este tipo de resultados con especial cuidado para destacar que el trabajo se ha realizado de manera sistemática.}