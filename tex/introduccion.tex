\chapter{Introducción} 
\label{ch:introduccion}

\makeatletter
En este documento se reportará el desarrollo de las prácticas en empresa llevadas a cabo bajo la supervisión de la UCLM
en la empresa \@business con el objetivo de convalidar la asignatura de \emph{Prácticas de Empresa Externas}.

Con estas prácticas se pretende afianzar y expandir las competencias impartidas por la universidad(Vease \ref{sec:contextualizacion}).



\section{Motivaciones}
\label{sec:motivaciones}

Enmarcando las prácticas en las preferencias y necesidades del alumno, se pretende enfocar esta actividad en la expansión de las competencias relacionadas con la programación y desarrollo de sistemas embebidos.

El material impartido sobre estas competencias en la UCLM es limitado dado su especialidad en fabricación y materiales y, aunque se tratan en algunas de las asignaturas \cite{deCastilla-LaMancha2024Sep}, su desarrollo se considera insuficiente para poder crear una carrera laboral relacionada con estas ramas.

Con estas prácticas se pretende corregir esta falta de formación para poder enfocar una carrera laboral para estos ámbitos.

\section{Zennio}
\label{sec:zennio}

La empresa elegida para la realización de las prácticas es \@business, una empresa especializada en el desarrollo, certificación y comercialización de productos domóticos KNX fundada en 2005 en la localidad Toledo.

Actualmente cuenta con más de 250 trabajadores y opera en 117 paises, teniendo su oficina y fábrica principal en Toledo, además de otras delegaciones repartidas por Europa y Oriente Medio.

En cuanto a la organización de la compañía, tenemos que está dirigida por un CEO (Juan Carlos Ciudad) y separada en 6 departamentos:

\alert{Corregir tabulación}

\begin{itemize}
    \item \emph{CLIPS:} Encargado de marketing y soporte al cliente.
    \item \emph{Desarrollo de Producto:} Encargado del hardware, la mecánica y la industralización.
    \item \emph{Ventas Corporativas:} Encargado de logística de ventas, oficina técnica y de mercados.
    \item \emph{Cadena de Suministro:} Encargado de compras, producción e ingeniería de producción.
    \item \emph{RRHH:} Encargado de relaciones laborales, organización y selección.
    \item \emph{Administración Financiera:} Encargado de finanzas, calidad de producto y IT.
\end{itemize}


\section{Contextualización de las prácticas en Zennio}
\label{sec:contextualizacion}

Las prácticas en Zennio han sido el resultado de varias condiciones globables y locales, pero se pueden simplifizar a dos sucesos principales.

En primer lugar tenemos el crecimiento de la compañía a finales de la década pasada, que fue detenido por la pandemia sucedida en 2020, seguido por una rápida expansión cuando sus efectos se redujeron.
En segundo lugar tenemos la falta de formación en programación y sistemas embebidos en la región de su sede.

Por tanto, durante la salida de la pandemia, Zennio tuvo que enfrentar una falta de personal que no era capaz de suplir facilmente debido a una falta de oferta. Como resultado de esta situción se creo el ZENNIO Talent Program, un programa desarrollado por la empresa enfocado a captar y formar personal por su cuenta.

La promoción del programa que nos concierce en este documento se centra en la formación en programación y sistemas embebidos, aunque también se traten más superficialmente otros departamentos para posibilitar el movimiento horizontal dentro de la empresa.

\makeatother