\chapter{Introducción} 
\label{ch:introduccion}

\avisoLocalizacionArchivo 

En este capítulo debes introducir el problema sin divagar, sin copiar de otros documentos y sin utilizar un lenguaje excesivamente técnico.  Tampoco utilices un lenguaje informal.  Este capítulo debería convencer al cliente de que el proyecto merece la pena.  Es decir, es un problema real y no está resuelto completamente.

Debe tenerse siempre presente que el cliente es el que paga.  En el TFG el que paga es el tribunal, en forma de calificación.  Así que a quien hay que convencer es a los miembros del tribunal.  El tribunal no lo conocerás a priori.  Por eso la memoria debe estar escrita para que la entienda alguien que no es especialista en el campo de aplicación.  Pero eso no implica que se toleren la falta de rigor o la falta de argumentación técnica.  Solo implica que los argumentos específicos hay que explicarlos o citar la fuente que los explica.

Redacta la introducción al final del TFG, cuando tengas elaborado el capítulo de antecedentes, los resultados y su discusión.  De esta forma podrás evitar repetir argumentos que ya están en esos capítulos.  La introducción debe introducir también el contexto en el que se desarrolla el TFG.  Divide el documento en secciones y subsecciones para organizar el contenido del capítulo.  Utiliza preferentemente frases cortas.


\section{Organización de la memoria} 
\label{sec:organizacion-memoria}

La organización de este documento responde a un documento científico-técnico. Se descompone en los siguientes capítulos.

\begin{description}
    \item[\autoref{ch:objetivos}] Enumera y justifica los objetivos del proyecto y establece los límites intrínsecos y extrínsecos de ejecución del TFG.
    \item[\autoref{ch:antecedentes}] Analiza los antecedentes y estado del arte en relación al tema del proyecto.
    \item[\autoref{ch:desarrollo}] Describe todo el proceso de desarrollo del TFG.  Esto incluye la metodología de trabajo empleada y las diferentes etapas o iteraciones que se han llevado a cabo.  No dudes en descomponer el capítulo en varios si aglutina demasiado material.
    \item[\autoref{ch:resultados}] Describe en detalle los resultados obtenidos y las pruebas realizadas. Discute los resultados en relación a los objetivos del proyecto.
    \item[\autoref{ch:conclusiones}] Recopila las principales conclusiones del proyecto y comenta las líneas de trabajo futuro, en caso de que se contemplen.
    \item[\deschyperlink{ch:anexos}{Anexos}] Complementan la información del cuerpo del documento con información técnica útil para reproducir los resultados, pero innecesaria para comprender en su totalidad el TFG realizado.
    \item[\deschyperlink{ch:bibliografia}{Bibliografía}] Recopila las referencias bibliográficas utilizadas en este documento.
\end{description}

\warning{La normativa de la EIIA no te obliga a usar una estructura fija.  Adáptala como quieras a tu TFG.  Puedes reordenar capítulos, mezclarlos, dividirlos, etc.  Por ejemplo, es frecuente poner unos antecedentes antes de los objetivos, que definen el contexto del TFG, y desarrollar el estado de la cuestión después de los objetivos.}

\warning{Al finalizar el resto de los capítulos revisa esta descripción del documento para que coincida con lo que realmente contiene la memoria.  Por ejemplo, es frecuente fusionar varios capítulos en uno cuando son muy pequeños.  También es frecuente lo contrario, dividir un capítulo en varios cuando es muy extenso.}

\section{Repositorio de información}
\label{sec:repositorio}

\warning{Es muy útil tanto para la ejecución como para la evaluación disponer de un repositorio para almacenar las sucesivas versiones del documento y de todo el material generado durante el proyecto.  La UCLM incorpora \href{https://github.com}{GitHub} como servicio institucional. Si no utilizas un repositorio quita esta sección.}

Todo el material generado durante la ejecución de este proyecto está disponible en el repositorio \thegitrepo{}.  El material incluye el código \LaTeX{} del presente documento, el código fuente de los programas realizados o modificados, y todos los datos generados en la evaluación de resultados.