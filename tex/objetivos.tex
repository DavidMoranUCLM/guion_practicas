\chapter{Objetivos}
\label{ch:objetivos}

El programa de formación es un intensivo de dos meses que pretende impartir todas las bases necesarias para facilitar la incorporación a la empresa en cualquiera de sus departamentos, por ello su curriculum es más variado de lo que unas practicas en sistemas embebidos podrían sugerir. El temario se ha organizado semanalnente de la siguiente manera:

\section*{Mayo 13-16: Softskills}
\label{sec:may13-16}

Durante esta semana se trabajarán habilidades dedicadas a las relaciones humanas, tanto con compañeros de trabajo como con superiores. Se busca mejorar la productividad, el trabajo en equipo y el ambiente laboral.

Se dividirá en:
\begin{enumerate}
    \item Onboarding: Presentación inicial del programa y compañeros
    \item Brainstorming: Mantener un flujo de ideas y extraer las mejores.
    \item Autoconocimiento DISC: Ser conscientes de cómo somos, cómo son los demás y cómo usar esa información para mejorar las relaciones laborales.
    \item Comunicación: Expresar ideas de forma clara y siendo influyente.
\end{enumerate}


\section*{Mayo 20-23: KNX, ETS y Catálogo}
\label{sec:may20-23}

Durante esta semana se introducirá el protocalo KNX que Zennio emplea para la comunicación entre sus dispositivos domóticos, así como el software ETS responsable de la configuración de los mismos. Además se presentarán los productos principales de la marca y sus respectivas categorias.

Se dividirá en:
\begin{enumerate}
    \item Introducción a KNX
    \item Control de luces
    \item Control de persianas
    \item Controles capacitivos y de habitaciones
    \item Paneles táctiles
    \item sensores
    \item Monitorización de energía
    \item Sistemas climáticos
    \item Aire acondicionado
    \item Sistemas de aerotermia
    \item Termostatos de Zennio
    \item Segmentación de zonas
    \item Sistemas de ventilación
    \item Soluciones FanCoil
    \item Soluciones de calefacción
    \item ALLinBOX
    \item Funciones lógicas
    \item Soluciones multimedia
    \item Sistemas KNX
\end{enumerate}

\section*{Mayo 27-29: Electrónica Básica}
\label{sec:may27-29}

En esta sección se expondrá desde cero conceptos básicos de la electrónica analógica-digital y su aplicaciones típicas en los diseños de producto en Zennio.

Se dividirá en:
 
\begin{enumerate}
    \item Fundamentos básicos
    \item Leyes
    \item Componentes básicos
    \item Sensores y actuadores
    \item Interfaces y comunicaciones
    \item Circuitería de Zennio
\end{enumerate}

\section*{Junio 3-13: Programación C}
\label{sec:jun3-13}

Programación en C está dividida en dos semanas puesto que es el temario principal de las prácticas.
Se impartirá desde cero, tratando todas las posibilidades que brinda el lenguaje de forma nativa.

Se dividirá en:

\begin{enumerate}
    \item Introducción a C
    \item helloWorld
    \item Variables
    \item Operadores y expresiones
    \item Control de flujo
    \item Funciones
    \item Punteros y arrays
    \item Estructuras, uniones y enumerados
    \item Constantes
    \item Librerías
    \item Memoria dinámica y listas
    \item Funciones públicas y privadas
    \item Punteros a funciones
\end{enumerate}

\section*{Junio 17-20: Código limpio y Patrones de diseño}
\label{sec:jun17-20}

Una vez entendido el lenguaje de programación C, se procederá a estudiar buenas prácticas y cómo organizar y estructurar de forma correcta un programa.

Se dividirá en:

\begin{enumerate}
    \item Introducción a código limpio
    \item Nomenclatura
    \item Funciones
    \item Comentarios
    \item Formato
    \item Pruebas unitarias
\end{enumerate}

\section*{Junio 24-27: Software embebido}
\label{sec:jun24-27}

Con las bases de programación aprendidas se aplicará a la programación de sistemas embebidos reales. Para esto se emplearán productos de la empresa y se tratará de diseñar funcionalidades básicas.

Se dividirá en:

\begin{enumerate}
    \item MCUXpresso
    \item Control de versiones SVN
    \item Software embebido
    \item Librerías y bloques funcionales
    \item Aplicaciones KNX
    \item Prácticas
\end{enumerate}

\section*{Julio 1-4: Pruebas y Testing}
\label{sec:jul1-4}

Por último se trabajará con hardware, aprendiendo a como planificar, diseñar, ejecutar y documentar pruebas que corroboren las especificaciones de un producto.

Se dividirá en:
\begin{enumerate}
    \item Principios del testing
    \item Tipologías de pruebas
    \item Diseño de pruebas
    \item Gestión de pruebas
    \item Riesgos
    \item Pruebas aplicadas en Zennio
    \item EITT
\end{enumerate}

\section*{Julio 5: Evaluación}
\label{sec:jul5}

El último día se reservará para evaluar el conocimiento adquirido durante la formación así como las capacidades de razonamiento generales.

\section*{Udemy}
\label{sec:udemy}

Además de las sesiones presenciales de lunes a jueves, se proporcionan cursos en la plataforma Udemy para reforzar los conceptos impartidos.

Los cursos se dividen en dos, uno centrado en el sistema operativo Linux y una selección varia de formaciones en softskills.

\section*{Presentación}
\label{sec:presentacion}

Para mostrar los conocimientos adquiridos durante las prácticas se realizará un trabajo grupal en el que se estudiará un posible producto que podría complementar con la filosofía de Zennio pero que no halla sido explorado aún.

El estudio debe contener:
\begin{itemize}
    \item Definición de necesidad
    \item Objetivo, alcance, recursos y riesgos
    \item Planificación
    \item Presupuesto
    \item Roles y responsabilidades
    \item Ejecución y monitorización de progreso
    \item Esquema del producto
    \item Presentación del proyecto
\end{itemize}

Ademas de contar con los roles:
\begin{itemize}
    \item Ing. de Soporte y Ventas
    \item Ing. Diseño Mecánico
    \item Ing. Diseño Hardware
    \item Ing. Comunicación y Protocolos
    \item Ing. Desarrollo SW embebido
    \item ing. Pruebas de Funcionalidad
    \item Ing. Pruebas de Integración y Hardware
\end{itemize}
